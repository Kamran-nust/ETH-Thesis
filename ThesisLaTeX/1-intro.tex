\chapter{Introduction}

This chapter mainly provides a brief introduction to the entire project. Section \ref{bckgrd} gives the background of the project and some basic concepts. Section \ref{prodef} defines the problems of this project to be solved. Section \ref{fcswrk} gives a brief introduction to the proposed new model. Section \ref{thsorg} gives the structure of this thesis for the convenience of readers.

\section{Background}\label{bckgrd}

Dummy text.

\subsection{Aerial Image}

航拍图像指的是……,(给例子)。它具有很大的应用。因为现实生活在不断更新,所展现的地图和也需要因此而更新。它能够对电子地图地图的绘制提供参考。它还有……的用途。这一节主要介绍毕设项目的背景。给出了什么是航拍图,什么是图像分割,以及展现我们所希望的到的几何图形。


\subsection{Image Segmentation}

Dummy text.

\subsection{Geometrical Shape}

Dummy text.

\section{Problem Definition}\label{prodef}




Instance segmentation is challenging because it requires the correct detection of all objects in an image while also precisely segmenting each instance. It therefore combines elements from the classical computer vision tasks of ob- ject detection, where the goal is to classify individual ob- jects and localize each using a bounding box, and semantic segmentation, where the goal is to classify each pixel into a fixed set of categories without differentiating object in- stances.
\section{Focus of This Work}\label{fcswrk}

Dummy text.

\section{Thesis Organization}\label{thsorg}

Dummy text.

Following common terminology, we use object detection to denote detection via bounding boxes, not masks, and semantic segmentation to denote per-pixel classification without differentiating instances. Yet we note that instance segmentation is both semantic and a form of detection.
