\begin{abstract}
\addcontentsline{toc}{chapter}{Abstract}
Deep learning methods have recently demonstrated remarkable achievements on image semantic segmentation. However, their architectures are still limited to pixel-wise labeling, which makes it hard to exploit high-level topology.

The goal of this project is to develop novel deep learning architectures for exploiting geometrical shapes of arbitrary structure. We want to depart from the standard paradigm that labels pixels but instead directly exploit and learn the geometry of the objects. In practice, we use dataset of aerial images, and aiming at extracting buildings' polygon shapes.

Two recent models, PolygonRNN and FPN, draw our attention. The former one can well exploit geometrical shape of single object in an image. The latter one performs well on the multi-scale object detection. Since our goal can be divided into two parts, objects detection and geometrical segmentation, the basic idea of the proposed solution would utilize these two models in two steps.

Specifically, in order to address the problem, we propose a new model, \modelnameshort\ (\modelnamelong), which is the integration of FPN (Feature Pyramid Network) and PolygonRNN. The model has three different versions, the two-step version, hybrid version and hybrid version with RoIAlign. We also introduce beam search to PolygonRNN in order to find the best polygon proposal.

Experiments show that our proposed model outperforms two previous works in the dataset of Chicago. The model can well localize each building within an aerial image, and for each building, it can well extract the building's geometrical shape.

\end{abstract}

\newpage

\renewcommand{\abstractname}{Acknowledgment}
\begin{abstract}
\addcontentsline{toc}{chapter}{Acknowledgment}
Foremost, I would like to express my sincere gratitude to Dr. Aurelien Lucchi and Dr. Jan Dirk Wegner for supervising my Master's thesis project, supporting me continuously and contributing many useful ideas. Their guidance helped me in all the time of discussing project and writing thesis, and I have learned a lot in the field of object detection and geometrical segmentation.

I would also like to thank Prof. Thomas Hofmann for providing me with the opportunity of this interesting project, as well as his suggestions about beam search. I would say doing Master's thesis project at Data Analytics Lab is an unforgettable experience for me.

Besides my supervisors, I would like to thank Tianhao Wei, a junior to me at Zhejiang University, for giving me many suggestions for the implementation details about PolygonRNN.

My sincere thanks also goes to my friends, Jingxuan He, Xiaojuan Wang, Canxi Chen, Jie Huang, Junlin Yao, and Renfei Liu, for all their helps, supports and companionship.

Last but not the least, I would like to thank my parents Haiyan Dai and Fasheng Li, for their spiritual supports and understandings throughout my studying life in Switzerland.

\end{abstract}